\section{Введение}
В современном мире информация представляет всё большую ценность.
Сложность её получения почти всегда окупается реализацией полученных знаний.
Немногие способны предоставить поистине качественную информацию.

Ей не характеры черты физического объекта,
	поэтому информация,
		являясь продуктом умственного труда,
	не способна быть защищённой средствами,
		которые применяются для защиты результатов,
			полученных другими видами деятельности.
Например: решение важного математического уравнения нельзя никому показывать,
	в то время как только что сшитый пиджак необходимо показать всем,
		чтобы он скорее был продан.
С другой стороны, решённое математическое уравнение невозможно полностью утратить,
	т.к. способ его решения останется вместе с решавшим уже до конца его осознанной жизни или предела памяти,
		а если украсть пиджак, то ничего не останется.
Такие очевидные различия между этими сущностями говорят о том,
	что методы защиты для них должны быть различными.

Иногда продукт интеллектуальной деятельности получить также сложно,
	как проделать неимоверную физическую работу,
а иногда и вовсе невозможно из-за различия мышления у людей.
