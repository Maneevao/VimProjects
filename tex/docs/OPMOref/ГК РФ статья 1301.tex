В случаях нарушения исключительного права на произведение автор или иной правообладатель наряду с использованием других применимых способов защиты и мер ответственности, установленных настоящим Кодексом ("статьи 1250", "1252" и "1253"), вправе в соответствии с "пунктом 3 статьи 1252" настоящего Кодекса требовать по своему выбору от нарушителя вместо возмещения убытков выплаты компенсации:
 
1) в размере от десяти тысяч рублей до пяти миллионов рублей, определяемом по усмотрению суда исходя из характера нарушения;
  
2) в двукратном размере стоимости контрафактных экземпляров произведения;
   
3) в двукратном размере стоимости права использования произведения, определяемой исходя из цены, которая при сравнимых обстоятельствах обычно взимается за правомерное использование произведения тем способом, который использовал нарушитель.
