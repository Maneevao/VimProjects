\section{Предмет защиты}
Сделаем предположение,
	что всё юридически оформленно,
		т.е. государство поможет защитить продукт вашей деятельности.
Не следует рассматривать,
	как юридически оформлять данную защиту,
		по причине существования отдельных документов,
			раскрывающих данную информацию.
В данной работе автор лишь перечислит основные методы защиты,
	а также те действия,
		которые предпримет государство для реализации права собственника.
\\
\\
\textit{ГК РФ Статья 1301. Ответственность за нарушение исключительного права на произведение:}\\

\input{"ГК РФ статья 1301"}
\\

Согласно кодексу, ответственность за нарушение смежных прав, патентного права точно такая же,
	как за наррушение авторских.
Таким образом, если нарушат ваши права,
	автор получит деньги за свою собственность,
	а также может удвоить их.
Это не такой и плохой вариант развития событий,
	поэтому настоятельно рекомендуется оформлять свои авторские права на объекты информационной собственности.
\\

Также в ГК выделяются такие права, как Исключительные,
	которые включены в интеллектуальные права,
	являющиеся имущественным правом.
\\
\\
\textit{Из ГК РФ Статья 1252. Защита исключительных прав:}\\

Т.к. текст статьи слишком большой для реферата,
	сделано решение о небольшом содержательном пересказе её положений.
Для более детального знакомства автор рекомендует прочесть первоисточник.
\\

Автор в праве заменить наказание на штраф, либо возмещение убытка(см. выше).
Изымаются не только объекты интеллектуальной собственности, но и средства,
	которыми данный объект был заполучен.
При конкуренции между несколькими объектами защиты,
	которые могут быть схожи,
	приоритет отдаётся тому,
		что было зарегистрировано раньше.
Также, если с помощью данной интеллектуальной собственности,
	была создана недобросовестная конкуренция,
		данная ситуация рассматривается также со стороны антимонопольным законодательством.
