\section{Арест и обвинение}
21 мая 2010 года Мэннинг, общаясь в чате с бывшим хакером Адрианом Ламо, рассказал ему, что это именно он передал WikiLeaks эту запись, а кроме них еще 260 тысяч секретных дипломатических сообщений, втайне от начальства скопировав их на CD-RW диск, ранее содержавший песни Леди Гага. Текст их переговоров в чате был впоследствии выложен в интернете. Ламо рассказал об этом властям США, и 29 мая Мэннинг был арестован и помещен в американскую тюрьму при базе "Кемп-Арифжан" на территории Кувейта, откуда был в июле перемещен в тюрьму на военной базе в Вирджинии. \cite{People} \cite{Manning} \\
Мэннинг был арестован по доносу Адриана Ламо в конце мая 2010 года. Без предъявления обвинения в течение двух месяцев он содержался в тюрьме при американской военной базе Кемп-Арифжан (Кувейт), затем в июле 2010 года был вывезен в США. \cite{Manning} \\
Неожиданно для многих, через некоторое время, Мэннинг признал свою вину по нескольким пунктам обвинения, исключая сговор с врагом. Юристы говорят, что таким путем он, скорее всего, хотел избежать трибунала и потенциального пожизненного заключения, но обвинение отказалось от досудебного соглашения и намерено доказать в суде, что ефрейтор Брэдли Мэннинг пытался нанести вред своей стране, открывая врагу американские секреты. \\
В обвинительной речи прокурора Джо Морроу Мэннинг предстал как военнослужащий из отдаленного гарнизона в Ираке, снедаемый желанием известности и удовлетворяющий свою страсть передачей секретных документов WikiLeaks. Как утверждает обвинение, Мэннинг держал с WikiLeaks постоянную связь, были случаи, когда документы появлялись на сайте Джулиана Ассанжа уже через два часа после отправки. \cite{Dream}
Военная прокуратура приняла признание Мэннинга лишь по одному из элементов обвинительного акта. По остальным она будет доказывать, что подсудимый не просто раскрыл гостайны, а сделал это злонамеренно в стремлении нанести урон интересам Соединенных Штатов и их союзников. Иными словами, что он был пособником врагов Америки. \cite{Sliv} \\
21 августа военный трибунал США приговорил Мэннинга к 35 годам лишения свободы за передачу секретных документов сайту Wikileaks, с правом досрочного освобождения через девять лет. Он был признан виновным по 20 пунктам обвинения – в том числе, в шпионаже (несмотря на отсутствие доказательств относительно намерений заниматься шпионажем, как и отсутствие доказательств того, что «утечка» информации нанесла реальный ущерб). \cite{Manning} \cite{Argument}
