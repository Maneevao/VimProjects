\section{Объект интеллектуальной собственности}
Если мы желаем защитить свой продукт,
	то делать это нужно согласованно с законом.
Данные вопросы уже освещены в ряде нормативно-правовых актов.
Ознакомимся с тем,
	с чем имеем дело,
	с помощью Гражданского Кодекса.\\

Согласно 4 части Гражданского Кодекса РФ
	раздела VII «Права на результаты интеллектуальной деятельности и средства индивидуализации».\cite{GKRF}\\

\underline{Интеллектуальная собственность} - список результатов интеллектуальной деятельности и средств индивидуализации,
	которым предоставляется правовая охрана.

\newpage
\underline{Формы, которые может принимать интеллектуальная собственность}:
\input{"Формы интеллектуальной собственности"}

Как видим, список внушительный,
	он позволит более детально разобраться в вопросе определения интеллектуальной собственности на законодательном уровне,
		а следовательно позволит приблизиться к пониманию того,
			что государство способно нам помочь защитить.
\\

Когда мы говорим о какой-либо собственности,
	необходимо, также помнить и про права,
		связанные с ней.
Право,
	относящееся к интеллектуальным объектам собственности,
	получило название Интеллектуальное Право.
Логично предположить, что,
	если мы рассмотрим данное право,
	то сможем приблизиться к пониманию того,
		в каких областях происходит защита.

\newpage
\input{"Виды интеллектуальных прав"}

Перечисленные права позволяют лучше понять,
	каким образом происходит защита интеллектуальной собственности.
