\section{Соответствие стандартам}
\subsection{Стандарт}
Стандартизация средства защиты информации проводилось на СВТ 2 уровня защищенности, руководствуясь документом с следующим названием:\\
\begin{center}
«Руководящий документ. Средства вычислительной техники. Защита от несанкционированного доступа к информации. Показатели защищенности от несанкционированного доступа к информации»
\cite{FSTECrdNSD}
\end{center}
Который был создан в ФСТЭК России.

\subsection{Показатели защищённости}
В соответствии с документом существуют следующие показатели защищённости:\\
\input{"Показатели защищённости"}
\subsection{Чему не удовлетворяет система}
Исходя из таблицы соответствия показателей защищённости классам защищенности
система не обязана обладать таким показателем защищённости как \textit{гарантия архитектуры}.
Что различает её от первого класса.\\

«2.7.16. Гарантии архитектуры.\\
КСЗ должен обладать механизмом, гарантирующим перехват диспетчером доступа всех обращений субъектов к объектам.»\cite{FSTECrdNSD}\\

