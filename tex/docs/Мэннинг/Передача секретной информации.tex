\section{Передача секретной информации}
В 2007 году Мэннинг поступил на службу в американскую армию. По долгу службы Мэннинг занимался первоначальным анализом разведывательной информации, связанной с Ираком, и у него был доступ к совершенно секретным сведениям. Во время работы он обнаружил засекреченную видеозапись расстрела людей с американского боевого вертолёта. В начале апреля 2010 года выложил её в интернете. Публикация этого видео вызвала большой общественный резонанс. \cite{Manning} \\
Багдадский авиаудар 12 июля 2007 года, также известный как Сопутствующее убийство — инцидент в иракской столице, когда в результате обстрела иракцев двумя вертолётами армии США погибли, по разным источникам, от 12 до более чем 18 человек, многие из которых не имели отношения к повстанческим группировкам. С вертолётов AH-64 Apache было произведено два обстрела 30-миллиметровыми пушками группы иракцев на улице. В результате погибли в том числе 2 репортёра Reuters; после чего ракетами AGM-114 Hellfire уничтожено здание, где среди погибших оказались женщины и дети. \\
В апреле 2010 года 39-минутная вертолётная видеозапись расстрела была опубликована сайтом WikiLeaks. \cite{Udar} Американские военные утверждают, что на видео не демонстрируется «ничего дурного», но ведь убийство безоружных журналистов – это военное преступление, а вся та грубость и жестокость, которую мы видим, дали нам ясное представление о том, что именно происходило в Ираке во время американской оккупации. \cite{Argument}