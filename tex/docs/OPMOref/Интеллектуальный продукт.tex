\section{Интеллектуальный продукт}
В результате работы в какой-либо области появляются новые знания,
	реализация которых способна помочь в решении определённых вопросов.
Такие знания называются интеллектуальным продуктом.
\cite{bibliotekar}

Интеллектуальный продукт способен принести пользу,
	а значит обладание этим продуктом желательно и не только для владельца.
При грамотной реализации полученных знаний,
	ситуация,
		над которой приходилось биться долгое время,
	способна разрешиться практически сама собой.

Люди не часто заинтересованы в поиске новой информации, знаний,
	это можно легко обосновать тем,
		что не всегда известно,
			что ищется.
Действительно, ведь множество открытий были сделаны почти случайно.
Не всегда можно предугадать,
	что получится,
		если работать в данном направлении.
Если вести счёт, то второй причиной было то,
	что учённые(люди, целенаправленно занимающиеся поиском новых знаний) не производят материальных благ,
		для которых в свою очередь нужна промежуточная стадия в виде исследования.
Не для всех очевидна польза исследований,
	но для многих очевидна польза производства.
Можно найти множество причин,
	почему интеллектуальный продукт так поздно стал рассматриваться как интеллектуальная собственность,
	что в свою очередь способствует в понимании в решении вопроса,
		как этот продукт защитить,
	но не ответит на него полностью.
