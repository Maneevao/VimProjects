\section{Общественная реакция}
Дело Мэннинга вызвало в США большой общественный резонанс, тем более что в прессу постоянно попадают данные, свидетельствующие о том, что условия его содержания под стражей далеки от цивилизованных. Те, кто видел Бредли после ареста, говорят, что опасаются за его психическое здоровье, так как он может не выдержать оказываемого на него давления и постоянных унижений. \\
Майк Гогулски, эмигрант из США, проживающий в Словакии, создал в июне 2010 года «Сеть поддержки Брэдли Мэннинга», проводившую митинги, а также протесты возле тюрьмы, где содержалась Мэннинг, а к августу 2012 более 12000 человек внесли в фонд Сети пожертвования общей суммой 650 000 долларов США, из которых 15 100 долларов поступило от WikiLeaks. \\
10 марта 2011 года, выступая в Массачусетском технологическом институте, пресс-секретарь Госдепа США Филипп Кроули сказал: «То, что происходит с Мэннингом, это глупо и контрпродуктивно. Не знаю, почему министерство обороны это делает». 14 марта чиновник подал в отставку. \\
В 2011 и 2012 годах кандидатура Мэннинга выдвигалась на получение Нобелевской премии мира, в последний раз её выдвинули Oklahoma Center for Conscience and Peace Research и три члена исландского парламента. \cite{Manning}