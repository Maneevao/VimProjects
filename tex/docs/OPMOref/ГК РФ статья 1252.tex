1. Защита исключительных прав на результаты интеллектуальной деятельности и на средства индивидуализации осуществляется, в частности, путем предъявления в порядке, предусмотренном настоящим Кодексом, требования:

1) о признании права - к лицу, которое отрицает или иным образом не признает право, нарушая тем самым интересы правообладателя;

2) о пресечении действий, нарушающих право или создающих угрозу его нарушения, - к лицу, совершающему такие действия или осуществляющему необходимые приготовления к ним, а также к иным лицам, которые могут пресечь такие действия;

3) о возмещении убытков - к лицу, неправомерно использовавшему результат интеллектуальной деятельности или средство индивидуализации без заключения соглашения с правообладателем (бездоговорное использование) либо иным образом нарушившему его исключительное право и причинившему ему ущерб, в том числе нарушившему его право на вознаграждение, предусмотренное "статьей 1245", "пунктом 3 статьи 1263" и "статьей 1326" настоящего Кодекса;

4) об изъятии материального носителя в соответствии с "пунктом 4" настоящей статьи - к его изготовителю, импортеру, хранителю, перевозчику, продавцу, иному распространителю, недобросовестному приобретателю;

5) о публикации решения суда о допущенном нарушении с указанием действительного правообладателя - к нарушителю исключительного права.

2. В порядке обеспечения иска по делу о нарушении исключительного права могут быть приняты соразмерные объему и характеру правонарушения обеспечительные меры, установленные процессуальным "законодательством", в том числе может быть наложен арест на материальные носители, оборудование и материалы, запрет на осуществление соответствующих действий в информационно-телекоммуникационных сетях, если в отношении таких материальных носителей, оборудования и материалов или в отношении таких действий выдвинуто предположение о нарушении исключительного права на результат интеллектуальной деятельности или на средство индивидуализации.

3. В случаях, предусмотренных настоящим Кодексом для отдельных видов результатов интеллектуальной деятельности или средств индивидуализации, при нарушении исключительного права правообладатель вправе вместо возмещения убытков требовать от нарушителя выплаты компенсации за нарушение указанного права. Компенсация подлежит взысканию при доказанности факта правонарушения. При этом правообладатель, обратившийся за защитой права, освобождается от доказывания размера причиненных ему убытков.
 
 Размер компенсации определяется судом в пределах, установленных настоящим "Кодексом", в зависимости от характера нарушения и иных обстоятельств дела с учетом требований разумности и справедливости.
  
  Если одним действием нарушены права на несколько результатов интеллектуальной деятельности или средств индивидуализации, размер компенсации определяется судом за каждый неправомерно используемый результат интеллектуальной деятельности или средство индивидуализации. При этом в случае, если права на соответствующие результаты или средства индивидуализации принадлежат одному правообладателю, общий размер компенсации за нарушение прав на них с учетом характера и последствий нарушения может быть снижен судом ниже пределов, установленных настоящим Кодексом, но не может составлять менее пятидесяти процентов суммы минимальных размеров всех компенсаций за допущенные нарушения.

4. В случае, когда изготовление, распространение или иное использование, а также импорт, перевозка или хранение материальных носителей, в которых выражены результат интеллектуальной деятельности или средство индивидуализации, приводят к нарушению исключительного права на такой результат или на такое средство, такие материальные носители считаются контрафактными и по решению суда подлежат изъятию из оборота и уничтожению без какой бы то ни было компенсации, если иные последствия не предусмотрены настоящим "Кодексом".
 
 5. Орудия, оборудование или иные средства, главным образом используемые или предназначенные для совершения нарушения исключительных прав на результаты интеллектуальной деятельности и на средства индивидуализации, по решению суда подлежат изъятию из оборота и уничтожению за счет нарушителя, если законом не предусмотрено их обращение в доход Российской Федерации.

 6. Если различные средства индивидуализации (фирменное наименование, товарный знак, знак обслуживания, коммерческое обозначение) оказываются тождественными или сходными до степени смешения и в результате такого тождества или сходства могут быть введены в заблуждение потребители и (или) контрагенты, преимущество имеет средство индивидуализации, исключительное право на которое возникло ранее, либо в случаях установления конвенционного или выставочного приоритета средство индивидуализации, которое имеет более ранний приоритет.
  
  Если средство индивидуализации и промышленный образец оказываются тождественными или сходными до степени смешения и в результате такого тождества или сходства могут быть введены в заблуждение потребители и (или) контрагенты, преимущество имеет средство индивидуализации или промышленный образец, исключительное право в отношении которого возникло ранее, либо в случаях установления конвенционного, выставочного или иного приоритета средство индивидуализации или промышленный образец, в отношении которого установлен более ранний приоритет.
   
   Обладатель такого исключительного права в порядке, установленном настоящим Кодексом, может требовать признания недействительным предоставления правовой охраны товарному знаку, знаку обслуживания, признания недействительным патента на промышленный образец либо полного или частичного запрета использования фирменного наименования или коммерческого обозначения.
    
    Для целей настоящего пункта под частичным запретом использования понимается:
    в отношении фирменного наименования запрет его использования в определенных видах деятельности;
    в отношении коммерческого обозначения запрет его использования в пределах определенной территории и (или) в определенных видах деятельности.

6.1. В случае, если одно нарушение исключительного права на результат интеллектуальной деятельности или средство индивидуализации совершено действиями нескольких лиц совместно, такие лица отвечают перед правообладателем солидарно.

7. В случаях, когда нарушение исключительного права на результат интеллектуальной деятельности или на средство индивидуализации признано в установленном порядке недобросовестной конкуренцией, защита нарушенного исключительного права может осуществляться как способами, предусмотренными настоящим Кодексом, так и в соответствии с антимонопольным "законодательством".
